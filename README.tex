\documentclass[a4paper,12pt]{article}
\usepackage[utf8]{inputenc}
\usepackage[T1]{fontenc}
\usepackage{lmodern}
\usepackage{geometry}
\usepackage{graphicx}
\usepackage{amsmath}
\usepackage{amsfonts}
\usepackage{amssymb}
\usepackage{hyperref}
\usepackage{listings}
\usepackage{xcolor}
\usepackage{enumitem}
\usepackage{fancyhdr}
\usepackage{lastpage}
\usepackage{titling}

% Page geometry
\geometry{margin=1in}

% Hyperlink setup
\hypersetup{
    colorlinks=true,
    linkcolor=blue,
    filecolor=magenta,
    urlcolor=cyan,
}

% Code listing style
\lstset{
    basicstyle=\ttfamily\small,
    keywordstyle=\color{blue},
    stringstyle=\color{red},
    commentstyle=\color{green!50!black},
    breaklines=true,
    showspaces=false,
    showstringspaces=false,
    frame=single,
    numbers=left,
    numberstyle=\tiny,
    tabsize=2,
}

% Header and footer
\pagestyle{fancy}
\fancyhf{}
\fancyhead[L]{\thetitle}
\fancyhead[R]{Page \thepage\ of \pageref{LastPage}}
\fancyfoot[C]{\today}

% Title and author
\title{Sports Reels: System Documentation}
\author{g-h-0-S-t}
\date{May 2025}

\begin{document}

% Title page
\maketitle
\begin{center}
    \vspace{1cm}
    \textbf{Version 1.0.0} \\
    \vspace{0.5cm}
    Deployed at: \url{https://sports-reels.onrender.com} \\
    Repository: \url{https://github.com/g-h-0-S-t/sports-reels}
\end{center}
\thispagestyle{empty}
\newpage

% Table of Contents
\tableofcontents
\newpage

\section{Introduction}
% Introducing the purpose of Sports Reels
Sports Reels is a web application that allows users to generate and view short video reels about sports celebrities. Users can input a celebrity's name, video title, description, and narration script to create a video montage of images sourced from Unsplash, accompanied by text-to-speech audio. Videos are stored in a GitHub repository and displayed in a scrollable, mobile-friendly interface. The application is built with Next.js for the frontend and backend, Python for video generation, and deployed on Render.

\subsection{Purpose}
% Explaining the application's objectives
The application enables fans to:
\begin{itemize}
    \item Generate personalized sports celebrity video reels.
    \item Search and view existing reels by celebrity name, title, or description.
    \item Enjoy a responsive, Instagram-like interface with auto-playing videos.
\end{itemize}

\subsection{Technology Stack}
% Listing technologies used
\begin{itemize}
    \item \textbf{Frontend}: Next.js 15.3.1, React 18.3.1, CSS (globals.css with Lato font).
    \item \textbf{Backend}: Next.js API routes, Node.js 18+, Python 3 (moviepy, gTTS).
    \item \textbf{Storage}: GitHub repository for videos and metadata (videos.json).
    \item \textbf{APIs}: Unsplash for images, GitHub API for file access.
    \item \textbf{Deployment}: Render (Node.js environment with Python support).
    \item \textbf{Dependencies}: node-fetch, dotenv, Pillow, numpy, imageio-ffmpeg, requests, tqdm.
\end{itemize}

\section{System Architecture}
% Describing the high-level architecture
Sports Reels follows a client-server architecture with a Next.js frontend and backend, integrated with a Python script for video generation. Videos and metadata are stored in a GitHub repository, accessed via API routes. The system is deployed as a single service on Render.

\subsection{Components}
% Detailing key components
\begin{itemize}
    \item \textbf{Frontend (pages/index.js)}: Renders the UI, handles user input, and displays videos.
    \item \textbf{Backend API Routes}:
        \begin{itemize}
            \item \texttt{/api/generate-video}: Triggers video generation and pushes to GitHub.
            \item \texttt{/api/proxy-json}: Fetches videos.json from GitHub.
            \item \texttt{/api/proxy-video}: Streams video files from GitHub.
            \item \texttt{/api/refresh-videos}: Refreshes video metadata.
        \end{itemize}
    \item \textbf{Python Script (generate\_videos.py)}: Generates videos using Unsplash images and gTTS audio.
    \item \textbf{GitHub Repository}: Stores videos (e.g., \texttt{videos/tiger-woods-history.mp4}) and metadata (\texttt{videos.json}).
    \item \textbf{Render Deployment}: Hosts the Node.js app and Python environment.
\end{itemize}

\subsection{Data Flow}
% Explaining data flow
\begin{enumerate}
    \item User submits a form with celebrity name, title, description, and script via \texttt{pages/index.js}.
    \item \texttt{/api/generate-video} clones the GitHub repo, runs \texttt{generate\_videos.py}, and pushes the video and updated \texttt{videos.json}.
    \item \texttt{generate\_videos.py} fetches images from Unsplash, generates audio with gTTS, and creates a 480p video with moviepy.
    \item \texttt{pages/index.js} polls \texttt{/api/proxy-video} to confirm video availability.
    \item \texttt{/api/refresh-videos} or \texttt{getStaticProps} fetches \texttt{videos.json} via \texttt{/api/proxy-json} to update the video list.
    \item Videos are streamed from GitHub via \texttt{/api/proxy-video} and displayed in the UI.
\end{enumerate}

\section{Frontend (pages/index.js)}
% Describing the frontend functionality
The frontend, implemented in \texttt{pages/index.js}, is a Next.js page that serves as the main interface. It uses React hooks for state management and IntersectionObserver for video autoplay.

\subsection{Features}
\begin{itemize}
    \item \textbf{Start Screen}: Displays a welcome message and "Start Reels" button.
    \item \textbf{Form}: Allows users to input celebrity name, title, description, and narration script. The video URL is auto-generated.
    \item \textbf{Search}: Filters videos by celebrity name, title, or description.
    \item \textbf{Reels Display}: Shows videos in a scrollable, full-screen layout with title and description overlays.
    \item \textbf{Loader}: Displays a spinning golf ball during video generation or refresh.
\end{itemize}

\subsection{State Management}
\begin{itemize}
    \item \texttt{videos}: Array of video objects from \texttt{videos.json}.
    \item \texttt{displayedVideos}: Filtered subset of \texttt{videos} for display.
    \item \texttt{isStarted}: Toggles between start screen and main UI.
    \item \texttt{formData}: Stores form inputs (celebrityName, title, description, customScript, videoUrl).
    \item \texttt{searchQuery}: Stores search input.
    \item \texttt{isGenerating}, \texttt{isRefreshing}: Control loader visibility.
    \item \texttt{error}: Displays error messages.
    \item \texttt{isFormActive}: Pauses videos when form is focused.
    \item \texttt{refreshKey}: Forces re-render of video list.
    \item \texttt{videoRefs}: References video elements for autoplay.
\end{itemize}

\subsection{Key Functions}
\begin{itemize}
    \item \texttt{getStaticProps}: Fetches \texttt{videos.json} at build time via \texttt{/api/proxy-json} with Incremental Static Regeneration (ISR, revalidate: 60s).
    \item \texttt{handleSubmit}: Sends form data to \texttt{/api/generate-video}, polls for video availability, and updates the video list.
    \item \texttt{pollVideo}: Polls \texttt{/api/proxy-video} (15 attempts, 5s intervals) to confirm video availability.
    \item \texttt{refreshVideos}: Fetches updated \texttt{videos.json} via \texttt{/api/refresh-videos}.
    \item \texttt{handleSearch}: Filters \texttt{displayedVideos} based on \texttt{searchQuery}.
    \item \texttt{fetchWithTimeout}: Handles HTTP requests with retries (2 attempts, 2s delay) and optional timeout (skipped for \texttt{/api/generate-video}).
\end{itemize}

\subsection{UI Behavior}
\begin{itemize}
    \item Videos autoplay when 50\% visible (via IntersectionObserver) and pause when out of view or form is active.
    \item The loader appears during video generation or refresh, with a spinning golf ball animation.
    \item Errors are displayed below the form (e.g., "Failed to generate video: ...").
    \item The UI is responsive, with mobile-friendly styles (globals.css).
\end{itemize}

\section{Backend (API Routes)}
% Describing API routes
The backend consists of Next.js API routes handling video generation, metadata fetching, and video streaming.

\subsection{/api/generate-video}
% Explaining video generation
\begin{itemize}
    \item \textbf{Method}: POST
    \item \textbf{Input}: JSON with \texttt{celebrityName}, \texttt{title}, \texttt{description}, \texttt{customScript}.
    \item \textbf{Process}:
        \begin{enumerate}
            \item Clones the GitHub repo to a temporary directory.
            \item Runs \texttt{generate\_videos.py} to create a 480p video.
            \item Updates \texttt{videos.json} with new video metadata.
            \item Commits and pushes changes to GitHub.
            \item Cleans up the temporary directory.
        \end{enumerate}
    \item \textbf{Output}: JSON with \texttt{videoUrl} and updated \texttt{videos} array.
    \item \textbf{Error Handling}: Returns 400 (missing fields), 500 (script or Git errors).
\end{itemize}

\subsection{/api/proxy-json}
% Explaining JSON proxy
\begin{itemize}
    \item \textbf{Method}: GET
    \item \textbf{Input}: Query parameter \texttt{url} (GitHub API URL for \texttt{videos.json}).
    \item \textbf{Process}: Fetches \texttt{videos.json} from GitHub with retries (5 attempts, 1s delay).
    \item \textbf{Output}: JSON content of \texttt{videos.json}.
    \item \textbf{Error Handling}: Returns 400 (missing URL), 500 (fetch errors).
\end{itemize}

\subsection{/api/proxy-video}
% Explaining video proxy
\begin{itemize}
    \item \textbf{Method}: GET
    \item \textbf{Input}: Query parameter \texttt{url} (GitHub raw video URL).
    \item \textbf{Process}: Streams video from GitHub with retries (5 attempts, 5s delay) and GitHub token authentication.
    \item \textbf{Output}: Video stream (Content-Type: video/mp4).
    \item \textbf{Error Handling}: Returns 400 (missing URL), 401 (invalid token), 403 (permissions), 404 (not found), 429 (rate limit), 500 (other errors).
\end{itemize}

\subsection{/api/refresh-videos}
% Explaining video refresh
\begin{itemize}
    \item \textbf{Method}: GET
    \item \textbf{Process}: Clones the GitHub repo, reads \texttt{videos.json}, and returns the videos array.
    \item \textbf{Output}: JSON with \texttt{videos} array.
    \item \textbf{Error Handling}: Returns 500 (clone or read errors).
\end{itemize}

\section{Python Video Generation (generate\_videos.py)}
% Describing the Python script
The \texttt{generate\_videos.py} script generates videos using Unsplash images and gTTS audio, optimized for Render's limited resources (512MB RAM, 0.1 CPU).

\subsection{Workflow}
\begin{enumerate}
    \item \textbf{Parse Arguments}: Accepts \texttt{--celebrity}, \texttt{--title}, \texttt{--description}, \texttt{--script}.
    \item \textbf{Download Images}: Fetches 10 images from Unsplash via API, resizes to 854x480.
    \item \textbf{Generate Audio}: Converts script to MP3 using gTTS.
    \item \textbf{Create Video}: Combines images (4.3s each, 24fps) with audio using moviepy, outputs 480p video (libx264, ultrafast preset).
    \item \textbf{Cleanup}: Removes temporary files.
\end{enumerate}

\subsection{Optimizations}
\begin{itemize}
    \item \textbf{Low Memory}: Resizes images to 480p, uses \texttt{ultrafast} preset, limits threads to 1.
    \item \textbf{Error Handling}: Logs errors to \texttt{generate\_videos.log}, verifies image integrity.
    \item \textbf{Temporary Directory}: Uses \texttt{tempfile} to manage intermediate files.
\end{itemize}

\section{Storage (GitHub Repository)}
% Explaining storage
Videos and metadata are stored in the GitHub repository \texttt{g-h-0-S-t/sports-reels-videos}:
\begin{itemize}
    \item \textbf{videos.json}: Contains an array of video objects (\texttt{id}, \texttt{celebrityName}, \texttt{title}, \texttt{description}, \texttt{videoUrl}).
    \item \textbf{videos/}: Stores MP4 files (e.g., \texttt{tiger-woods-history.mp4}).
    \item \textbf{Access}: Uses GitHub Personal Access Token (GITHUB\_TOKEN) with \texttt{repo} or \texttt{public\_repo} scope.
\end{itemize}

\section{Deployment (Render)}
% Explaining deployment
The application is deployed on Render as a web service:
\begin{itemize}
    \item \textbf{Environment}: Node.js with Python 3 support.
    \item \textbf{Build}: Installs Node (npm install) and Python dependencies (pip3 install -r requirements.txt), runs Next.js build.
    \item \textbf{Start}: Runs \texttt{npm run start}.
    \item \textbf{Environment Variables}:
        \begin{itemize}
            \item \texttt{NODE\_ENV}: production
            \item \texttt{NEXT\_PUBLIC\_APP\_URL}: \url{https://sports-reels.onrender.com}
            \item \texttt{GITHUB\_TOKEN}: GitHub PAT
            \item \texttt{UNSPLASH\_ACCESS\_KEY}: Unsplash API key
        \end{itemize}
    \item \textbf{Configuration}: \texttt{render.yaml} defines the service.
\end{itemize}

\section{Workflow Example}
% Providing a user workflow
\begin{enumerate}
    \item User visits \url{https://sports-reels.onrender.com}, clicks "Start Reels".
    \item User enters:
        \begin{itemize}
            \item Celebrity Name: Tiger Woods
            \item Title: Tiger Woods History
            \item Description: A short history of Tiger Woods
            \item Script: Eldrick Tont Tiger Woods, born December 30, 1975, is an American professional golfer...
        \end{itemize}
    \item Form submits to \texttt{/api/generate-video}, which:
        \begin{itemize}
            \item Clones the GitHub repo.
            \item Runs \texttt{generate\_videos.py} to create \texttt{videos/tiger-woods-history.mp4}.
            \item Updates \texttt{videos.json}.
            \item Pushes to GitHub.
        \end{itemize}
    \item UI polls \texttt{/api/proxy-video} until the video is available.
    \item Video appears in the reels container, auto-plays when visible.
    \item User searches "Tiger" to filter videos.
\end{enumerate}

\section{Error Handling}
% Explaining error handling
\begin{itemize}
    \item \textbf{Frontend}: Displays errors (e.g., "Failed to generate video: ...") below the form.
    \item \textbf{Backend}: Returns HTTP status codes (400, 401, 403, 404, 429, 500) with error messages.
    \item \textbf{Python}: Logs errors to \texttt{generate\_videos.log}, raises exceptions for failures.
    \item \textbf{Retries}: \texttt{/api/proxy-json} (5 attempts, 1s delay), \texttt{/api/proxy-video} (5 attempts, 5s delay), \texttt{fetchWithTimeout} (2 attempts, 2s delay).
\end{itemize}

\section{Maintenance}
% Providing maintenance tips
\begin{itemize}
    \item \textbf{Monitor Logs}: Check Render logs and \texttt{generate\_videos.log} for errors.
    \item \textbf{Update Dependencies}: Regularly update \texttt{package.json} and \texttt{requirements.txt}.
    \item \textbf{GitHub Token}: Rotate \texttt{GITHUB\_TOKEN} periodically, ensure correct scopes.
    \item \textbf{Unsplash API}: Monitor rate limits (50 requests/hour).
    \item \textbf{Render Resources}: Upgrade from free tier if performance issues arise.
\end{itemize}

\section{Conclusion}
% Summarizing the application
Sports Reels is a user-friendly application for creating and viewing sports celebrity video reels. Its integration of Next.js, Python, GitHub, and Render provides a scalable solution for video generation and display. The codebase is optimized for low-resource environments, with robust error handling and a responsive UI.

\end{document}